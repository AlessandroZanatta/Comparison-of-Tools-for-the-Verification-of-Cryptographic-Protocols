% Section 1
% 2021-08-19
% Alessandro Zanatta

\section{Introduction}
\label{section:introduction}

Security protocols are used everyday by billions of users and applications to guarantee a certain degree of security and privacy over communications happening on the (insecure) Internet (MTProto2.0 \cite{Telegram-MTProto2.0}, SSH \cite{rfc4251} and TLSv1.3 \cite{TLSv1.3_specs} protocols are just a few famous examples). Although, there is a catch: designing such protocols has been proven to be extremely error prone. As an example, consider the Needham-Schroeder public-key protocol \cite{NSPK}, which has been believed to be secure for almost 20 years - before a fatal flaw was found and corrected \cite{NSPK_LoweGavin}.

Given the importance of the correctness of such protocols and the difficulties for designers to ensure it, it has been necessary to \textit{formally} prove the absence of security vulnerabilities. Towards this aim, a set of tools has been developed to assist the designing process of a new protocol.

\subsection{Classification of tools}

Two classes of models employed by these tools can be identified: symbolic and computational. As the tools we are going to examine exploit the symbolic model, we will not explore the computational model (see \cite{ReconcilingComputationalSymbolic, SymbolicComputationalBlanchet, 10.1007/978-3-540-31987-0_12} for further readings).

Let us briefly examine the symbolic model. This model assumes a Dolev-Yao \cite{Dolev-Yao} attacker. Moreover, cryptographic primitives are considered as black-box and are represented using function symbols, the messages are terms and the adversary can only use defined primitives. An important aspect to note of this model is that it assumes \textbf{perfect cryptography}. As an example, consider the case in which there are two function symbols (\textit{enc} and \textit{dec}, used to encrypt and decrypt), a message \textit{m} and a key \textit{k} and the following equality is defined:

\begin{equation}
    \label{eq:perfect-crypto}
    \mbox{dec}\left(\mbox{enc}\left(m, k\right), k\right) = m
\end{equation}

Following from \cref{eq:perfect-crypto} $-$ and considering the perfect cryptography assumption $-$ it is possible to decrypt $\mbox{enc}\left(m, k\right)$ if and only if \textit{k} is known \cite{SymbolicComputationalBlanchet}.

In the next sections, we will have a look at the internal reasoning, at some examples and at the comparison of three automatic tools: \textbf{Tamarin-prover, Proverif and Verifpal}. Notice that these tools work with an unbounded number of sessions, making the problem undecidable \cite{SymbolicComputationalBlanchet}.