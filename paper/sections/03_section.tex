% Section 3
% 2021-08-19
% Alessandro Zanatta

\section{Case study}
\label{section:case-study}

The full models of the following examples are freely available on github \cite{CaseStudies}.

\subsection{Diffie-Hellman key exchange}

We will examine three different models for the Diffie-Hellman key exchange protocol: anonymous, ephemeral and ephemeral with Perfect Forward Secrecy. This allows us to explore many advanced features and modelling techniques of each tool. \Cref{fig:dh-key-exchange} shows a schematic representation of the protocol.

In the anonymous version we model an un-authenticated Diffie-Hellman, where no security property can hold as the adversary is free to perform a man-in-the-middle attack \cite{MITM-DH}.

In the ephemeral version we model a client-server Diffie-Hellman exchange in which the server's half key is authenticated (e.g. by an X.509 certificate). The server is willing to execute the protocol with any client, but the server is always authenticated.

Finally, we model a Diffie-Hellman with post-compromise of ephemeral keys. This is how Perfect Forward Secrecy \cite{PFS} is usually tested in the symbolic model. Of course, secrecy of the exchanged messages will not hold.

\begin{figure}[t]
    \setmscoptions
    \begin{msc}{}
    \setmscscale{.7} 
    
    \declinst{client}{}{Client}
    \declinst{server}{}{Server}
    
    \action*{\parbox{3.5cm}{\centering
        Knows $g, p$\\
        $c \in \Z$\\
        $g_c := \modexp{g}{c}{p}$
    }}{client}

    \nextlevel[5]
    \mess{$g_c$}{client}{server}
    \nextlevel

    \action*{\parbox{3.5cm}{\centering
        Knows $g, p$\\
        $s \in \Z$\\
        $g_s := \modexp{g}{s}{p}$\\
        $\skey{c} := \modexp{g_c}{s}{p}$
    }}{server}

    \nextlevel[6]
    \mess{$g_s$}{server}{client}
    \nextlevel

    \action*{\parbox{3.5cm}{\centering
        $\key{cs} := \modexp{g_s}{c}{p}$
    }}{client}
    \nextlevel

    \end{msc}
    \centering
    \caption{Diffie-Hellman key exchange protocol}
    \label{fig:dh-key-exchange}
\end{figure}

\subsubsection{Implementation notes}

We will describe implementation details worth of note in the next paragraphs. Notice that we will use an incremental approach, describing only what was \textit{changed} between different versions. 

\paragraph{Anonymous} The implementation of the anonymous version is straightforward. We use two particular constructs for Proverif and Tamarin to store keys: tables and persistent facts, respectively. Both serve the same scope: store some permanent information (which is not, by default, available to the attacker). The first supports two operations: \textit{insert} and \textit{get}. The latter is, in essence, a \textit{fact} with the additional property of never being removed from the global state.

\paragraph{Ephemeral} In this version, the server half key needs to be authenticated\footnote{An additional requirement is that, while the half key cannot be mutated by the attacker, it must be available to it.}. In Verifpal there is a single (simple) way: guarded variables. Guarded variables (variables between square brackets) are not mutable by the attacker (but known by it). In Tamarin we model this with a simple private channel ruled by a passive attacker. This is done with the following multiset rewriting rule:


\lstset{language=tamarin}
\begin{lstlisting}
rule CertificateExchange:
    [ CertificateOut(x) ] --> [ CertificateIn(x), Out(x) ]
\end{lstlisting}

Finally, in Proverif we model this with a table, additionally making sure to output everything we insert into it. We could have also used a private channel, but that makes the verification of certain queries yield an inconclusive result.

\paragraph{Post-compromise} In tamarin, we define the following rule (for both client and server):

\lstset{language=tamarin}
\begin{lstlisting}
rule RevealClientEphemeralKey:
    [ ClientEphemeralKey(c) ]
  --[ RevealedClientEphemeralKey(c) ]->
    [ Out(c) ]
\end{lstlisting}

This actually models a \textit{generic} compromise rule. We can exploit timepoints in security properties to assert that the compromise must happen \textbf{after} some other event, making this a post-compromise.

In Proverif, we define a process macro which simply outputs the ephemeral key in phase 1:

\lstset{language=proverif}
\begin{lstlisting}
let PostRevealClientEphemeralKey =
    phase 1;

    (* Get the client's ephemeral key and output it *)
    get ClientEphemeralKeyTable(c) in
    out(io, c);
    event PostRevealedClientEphemeralKeyTable(c);
    0.
\end{lstlisting}

\lstset{language=verifpal}
Modeling post-compromise in Verifpal requires using phases again. We can then use Verifpal's built-in construct for revealing terms: \lstinline{leaks x}.  
\begin{lstlisting}
phase [1]

principal Client [
    leaks c
]

principal Server [
    leaks s
]
\end{lstlisting}

\subsection{Needham-Schroeder Public Key protocol}
\label{sec:NSPK}

\begin{figure}[t]
    \setmscoptions
    \begin{msc}{}
    \setmscscale{.7} 
    
    \declinst{alice}{}{Alice}
    \declinst{bob}{}{Bob}
    
    \action*{\parbox{3.5cm}{\centering
        Knows $\skey{A}, \pkey{A}$\\
        Knows $\pkey{B}$
    }}{alice}

    \action*{\parbox{3.5cm}{\centering
        Knows $\skey{B}, \pkey{B}$\\
        Knows $\pkey{A}$
    }}{bob}
    \nextlevel[4]

    \action*{\parbox{3.5cm}{\centering
        Generates $N_A$
    }}{alice}

    \nextlevel[3]    
    \mess{$\enc{N_A, A}{\pkey{B}}$}{alice}{bob}
    \nextlevel


    \action*{\parbox{3.5cm}{\centering
        Generates $N_B$
    }}{bob}

    \nextlevel[3]
    \mess{$\enc{N_A, N_B}{\pkey{A}}$}{bob}{alice}
    \nextlevel[2]
    \mess{$\enc{N_B}{\pkey{B}}$}{alice}{bob}

    \end{msc}

    \centering
    \caption{Simplified Needham-Schroeder Public Key protocol}
    \label{fig:NSPK}
\end{figure}

The next case-study is the Needham-Schroeder Public Key protocol. Two versions are considered: the flawed version and the fixed one (by Lowe \cite{NSPK_LoweGavin}). A schematic representation of the flawed protocol is shown in \cref{fig:NSPK}, which assumes that clients already know each others public keys. In the fixed version, message 2 is changed from $\enc{N_A, N_B}{\pkey{A}}$ to $\enc{B, N_A, N_B}{\pkey{A}}$.

\subsubsection{Implementation notes}

In order to re-discover the attack, we model the protocol in a way that allows the attacker to decide whom the honest initiator executes the protocol with. The responder, however, only wishes to engage with the honest initiator.


\paragraph{Flawed} 

In Proverif, to allow the behaviour described above, we let the attacker send the principals to the initiator at the start of its macro:

\lstset{language=proverif}
\begin{lstlisting}
let Initiator() =
(* Attacker sends communicating parties *)
in(io, (X: Principal, rUser: Principal));

(* We restrict them: X must be honest and rUser must not be X *)
let iUser = isHonest(X, Alice, Bob) in
if iUser <> rUser then (* Protocol starts here *)
\end{lstlisting}

In Tamarin we follow a very similar approach, but we define the initiator to be always Alice (i.e. the attacker chooses \textit{only} the responder). Using the same strategy of Proverif would lead to a very complex protocol.

In Verifpal the implementation is straightforward, with a single caveat: when pre-sharing public keys we need to guard only Alice key. This allows the attacker to mutate Bob's key to its own.

\lstset{language=verifpal}
\begin{lstlisting} 
Alice -> Bob: [K_PA]
Bob -> Alice: K_PB
\end{lstlisting}


\paragraph{Fixed} No changes to the models were applied but the fix to the Needham-Schroeder Public Key protocol described at the start of \cref{sec:NSPK}.